\documentclass[sigconf]{acmart}
\usepackage{times}
\usepackage[utf8]{inputenc}

\begin{document}

\title{Technical Problems Addressed by "A First Look at the Crypto-Mining Malware Ecosystem: A Decade of Unrestricted Wealth"}

\author{Firstname Lastname}
\affiliation{%
  \institution{M12345678} 
}

\begin{abstract}
Crypto-mining malware is a type of malware that steals computing resources from victims to mine cryptocurrency for cybercriminals. This paper, "A First Look at the Crypto-Mining Malware Ecosystem: A Decade of Unrestricted Wealth", is one of the most comprehensive studies of crypto-mining malware to date. The paper analyzes approximately 4.5 million malware samples (1.2 million malicious miners) over a period of twelve years from 2007 to 2019. The authors found that crypto-mining malware is a significant threat to computer users, and that it has generated millions of dollars in illicit profits for cybercriminals. The paper also provides valuable insights into the infrastructure and economics of the crypto-mining malware ecosystem.

This summary will describe the technical problems addressed by the paper. The paper addresses the following technical problems:

    The lack of understanding of the crypto-mining malware ecosystem.
    The difficulty of detecting and removing crypto-mining malware.
    The need for better methods of preventing crypto-mining malware infections.

The paper provides valuable insights into these problems, and it offers several recommendations for future research.
\end{abstract}

\section{Introduction}

Crypto-mining malware is a type of malware that steals computing resources from victims to mine cryptocurrency for cybercriminals. This can have a significant impact on victims, as it can slow down their computers and increase their electricity bills. In some cases, crypto-mining malware can even damage victims' computers.

The paper, "A First Look at the Crypto-Mining Malware Ecosystem: A Decade of Unrestricted Wealth", is one of the most comprehensive studies of crypto-mining malware to date. The paper analyzes approximately 4.5 million malware samples (1.2 million malicious miners) over a period of twelve years from 2007 to 2019. The authors found that crypto-mining malware is a significant threat to computer users, and that it has generated millions of dollars in illicit profits for cybercriminals. The paper also provides valuable insights into the infrastructure and economics of the crypto-mining malware ecosystem.

This summary will describe the technical problems addressed by the paper. The paper addresses the following technical problems:

    The lack of understanding of the crypto-mining malware ecosystem.
    The difficulty of detecting and removing crypto-mining malware.
    The need for better methods of preventing crypto-mining malware infections.

The paper provides valuable insights into these problems, and it offers several recommendations for future research.

\section{Technical Problems}

\subsection{Lack of Understanding}

There is a lack of understanding of the crypto-mining malware ecosystem. This is due to the fact that crypto-mining malware is a relatively new threat, and it is constantly evolving. As a result, security researchers and antivirus companies are struggling to keep up with the latest trends in crypto-mining malware.

The paper addresses this problem by providing a comprehensive overview of the crypto-mining malware ecosystem. The paper describes the different types of crypto-mining malware, how they work, and how they are distributed. The paper also provides insights into the infrastructure and economics of the crypto-mining malware ecosystem.

\subsection{Detection and Removal}

Crypto-mining malware is difficult to detect and remove. This is because crypto-mining malware is often designed to hide its activity from antivirus software and other security tools. Additionally, crypto-mining malware can be very persistent, and it can be difficult to remove completely.

The paper addresses this problem by describing several methods for detecting and removing crypto-mining malware. The paper also provides recommendations for improving the detection and removal of crypto-mining malware.

\subsection{Prevention}

There is a need for better methods of preventing crypto-mining malware infections. This is because crypto-mining malware is becoming increasingly common, and it is causing significant harm to computer users.

The paper addresses this problem by describing several methods for preventing crypto-mining malware infections. The paper also provides recommendations for improving the prevention of crypto-mining malware infections.

\section{Recommendations}

The paper offers several recommendations for future research. These recommendations include:

    Developing better methods for understanding the crypto-mining malware ecosystem.
    Developing more effective methods for detecting and removing crypto-mining malware.
    Developing better methods for preventing crypto-mining malware infections.

These recommendations can help to improve the security of computer users and protect them
\end{document}
